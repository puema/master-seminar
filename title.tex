\titlehead{
\raggedleft
\includegraphics[scale=0.7]{\figdir/HS_Logo_aktuell_CMYK.eps}
}
%
%\vspace*{-1cm}
\subject{Masterseminar}

\title{
Softwarevisualisierung für Virtual Reality
}

\author{
Marcel Pütz\\
Fakultät für Informatik\\
{\small In Zusammenarbeit mit der QAware GmbH}
}

\date{SS 2017}

\maketitle

\begin{abstract}
Visualisierung spielt in vielen Bereichen der Wissenschaft eine wichtige Rolle. In der Informatik ist die Softwarevisualisierung bei der Größe und Komplexität zeitgemäßer Softwaresysteme eine hilfreiche Unterstützung für die Entwicklung, Exploration und Analyse von Software.

Mit dem zunehmend Einzug haltenden Medium der Virtuellen Realität, bekommt die Softwarevisualisierung neue Möglichkeiten und besseren Zugang zu dreidimensionalen Modellen der Visualisierung.

In dieser Arbeit sollen Modelle der dreidimensionalen Softwarevisualisierung untersucht werden. Dazu wird mithilfe den Ergebnissen einer Umfrage erarbeitet, was eine Softwarevisualisierung überhaupt erreichen soll. Anhand der daraus abgeleiteten Anforderungen, können die verschiedene Modelle miteinander verglichen werden. Die mögliche Realisierung der Modelle für die Microsoft HoloLens spielt in dieser Arbeit ebenfalls eine Rolle.

Es wird besonders auf die Metapher der Software-Stadt eingegangen, aber auch kreativ nach neuen, alternativen Metaphern gesucht. Am Schluss der Arbeit wird mithilfe des Gegenüberstellung und der Bewertung der behandelten Modelle eine Empfehlung für die Realisierung einer Softwarevisualisierung für die HoloLens gegeben.

\end{abstract}

\clearpage

\tableofcontents

\addtokomafont{caption}{\small}
